\documentclass[LPSC_Labo03_SDeriaz]{subfiles}


\begin{document}
\section{Introduction}
Le but de se laboratoire est d'implémenter un affichage de la fractale de Mandelbrot sur FPGA.
\begin{figure}[H]
\centering
\includegraphics[width=10cm]{mandelbrot.png}
\caption{Fractale de Mandelbrot}
\end{figure}
Pour obtenir la fractale, il faut évaluer l'équation \ref{eq_mandelbrot} avec comme point de départ $C$ (dans le plan complexe).
\begin{equation}
z_{k+1}=z_{k}^2+C
\label{eq_mandelbrot}
\end{equation}
Après chaque itération, on évalue si la norme euclidienne du nombre complexe $z_{k+1}$ dépasse un rayon donné $R=2$. Le nombre d'itérations réalisées jusqu'ici constitue la valeur associée à chaque pixel.
\subsection{Couleurs}
Pour passer d'un nombre d'itérations (de 0 à 100), on utilise une lookup table pour affecter une couleur à chaque valeur d'itération (et rendre la fractale plus intéressante à regarder)
\end{document}