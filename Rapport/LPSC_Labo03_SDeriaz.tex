\documentclass[]{article}


\usepackage[french]{babel}
\usepackage[utf8]{inputenc}
\usepackage{lmodern}

\usepackage{graphicx}
\usepackage{fancyhdr}
\usepackage[margin=2.5cm,includehead]{geometry}
\usepackage{lastpage}
\usepackage[dvipsnames]{xcolor}
\usepackage{subfiles}
\usepackage[bottom]{footmisc}
\usepackage{float}
\usepackage{epsfig}
%\usepackage{amsmath}
\usepackage{siunitx}
\usepackage{mathtools}
\usepackage{tcolorbox}

\usepackage{hyperref}
\hypersetup{
    colorlinks,
    citecolor=black,
    filecolor=black,
    linkcolor=black,
    urlcolor=black
}

\DeclareMathOperator{\Div}{div}
\DeclareMathOperator{\Rot}{rot}
\DeclareUnicodeCharacter{2009}{\,} 

\def\screenshotWidth{0.9}


% Settings
\newcommand{\Author}{Sébastien Deriaz}
\newcommand{\professor}{Fabien Vannel}
\newcommand{\assistant}{Joachim Schmidt}
\newcommand{\cours}{Logique programmable pour systèmes complexes et performants (LPSC)}
\newcommand{\titre}{Laboratoire Mandelbrot}

\newcommand{\qbox}[1]{\begin{tcolorbox}\emph{#1}\end{tcolorbox}}

%Figure
\newcommand{\figc}[3]{%
  \begin{figure}[H]
    \centering
    \includegraphics[width=#1\textwidth]{Figure/#2.png}
    \caption[caption]{#3}
    \label{fig:#2}
  \end{figure}}
  
\def\doubleunderline#1{\underline{\underline{#1}}}

\DeclareUnicodeCharacter{202F}{\,}

\usepackage[outdir=./]{epstopdf}

% Other packages

% Page config
\fancyhead[L]{\includegraphics[height=8mm]{mse-full-cropped.pdf}}

\pagestyle{fancy}
\fancyfoot[C]{\thepage\ / \pageref{LastPage}}

\begin{document}
\thispagestyle{empty}

\begin{center}
\includegraphics[height=8mm]{mse-full-cropped.pdf}
\hfill
\includegraphics[height=8mm]{HES-SO_logo_Pantone.pdf}
\end{center}
\vfill
\begin{center}
\Huge \titre\\
\end{center}
\begin{center}
\large
\begin{tabular}{lp{7.5cm}}
Département : & EIE\\
Unité d'enseignement : & \cours
\end{tabular}
\end{center}
\vfill
\begin{center}
\large
\begin{tabular}{ll}
\Large Auteur & \Large \Author\\
Professeur & \professor\\ 
Assistant & \assistant\\
Date & \today
\end{tabular}
\end{center}
\vfill

\pagebreak
\tableofcontents
\pagebreak
\subfile{introduction}
\pagebreak
\subfile{analyse}
\pagebreak
\subfile{realisation}
\pagebreak
\subfile{simulations}
\pagebreak
\subfile{conclusion}

\begin{flushright}
\begin{tabular}{r m{3cm}}
Lausanne, le \today & \\
\Author & \\
\end{tabular}
\end{flushright}

\pagebreak

\end{document}